\documentclass[notes,compress]{beamer}

\usepackage[english, russian]{babel}
\usepackage[utf8x]{inputenc}
\usepackage{amsmath,amssymb}
\usepackage{graphicx}

\graphicspath{ {./imgs/} }
\DeclareGraphicsExtensions{.jpg,.png}

\mode<presentation>
{
  \usepackage[bars]{beamerthemetree}
  \beamertemplatetransparentcovereddynamic
  \beamertemplateballitem
}

\usetheme{Antibes}
\usecolortheme{dolphin}

\usepackage[T1]{fontenc}
\beamertemplatetransparentcovered




\title
  	{{Анализ контуров.}}
 

 %\subject{Theoretical Computer Science}
 \date{Екатеринбург,~2015}


 \begin{document}
    \frame{\titlepage}
    
    \begin{frame}
        \frametitle{Введение}
        \uncover<+-> {Рассматриваемые в данной лекции контуры - это границы объектов.
        То есть замкнутые кривые без самопересечений.}
    \end{frame}
    
    \begin{frame}
        \begin{columns}[T]
            
            \begin{column}{.4\textwidth}
                \begin{columns}
                    \begin{column}{\textwidth}
                        \visible<2->{\includegraphics[width=\textwidth]{FirstPixel}}
                    \end{column}
                \end{columns}
                
                \begin{columns}
                    \begin{column}{\textwidth}
                        \visible<1->{\includegraphics[width=\textwidth]{contour_def}}
                    \end{column}
                \end{columns}
        	    
            \end{column}            
            
            \begin{column}{.1\textwidth}
        	    \visible<2->{\includegraphics[width=\textwidth]{ContourCoords}}
            \end{column}
            
            {\begin{column}{.5\textwidth}
                \visible<1->{Интересующие нас объекты располагаются на пиксельной сетке, значит контур - последовательность пикселей.}
                \newline
                \newline
        	    \visible<2->{В дальнейшем будет удобно представлять каждый пиксель контура не через абсолютные координаты,
                            а как сдвиг относительно предыдущего}
            \end{column}}
           
        \end{columns}
	   
       
    \end{frame}
    
    \begin{frame}

    \end{frame}
    
    \begin{frame}

    \end{frame}
    
    
    \begin{frame}
        \frametitle{ввдение}
        {\Huge текст}
        \begin{itemize}
          \item<+-> first
          \item<+-> second
          \item<+-> third

            \begin{enumerate}
              \item<1-> 1
              \item<1-2> \alert<5-> {2}
              \item<3-> third
            \end{enumerate}
        \end{itemize}
        \uncover<4-> {hi}
    \end{frame}

    \begin{frame}
        \includegraphics[width=.8\textwidth]{firstImage}
    \end{frame}

    \begin{frame}
    	\begin{columns}[T]
        	\begin{column}{.5\textwidth}
            	Your textblock
           	
        	\end{column}
        	\begin{column}{.5\textwidth}
            	\begin{block}{Your image}
                	\includegraphics[width=\textwidth]{firstImage}
            	\end{block}
        	\end{column}
    	\end{columns}
	\end{frame}

    \begin{frame}
        Improved integral $\int\limits_{a}^{b} x^2 dx$ inside text
    \end{frame}

\end{document} 