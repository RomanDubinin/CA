\documentclass[notes,compress]{beamer}

\usepackage[english, russian]{babel}
\usepackage[utf8x]{inputenc}
\usepackage{amsmath,amssymb}
\usepackage{graphicx}

\graphicspath{ {./imgs/} }
\DeclareGraphicsExtensions{.jpg,.png}

\mode<presentation>
{
  \usepackage[bars]{beamerthemetree}
  \beamertemplatetransparentcovereddynamic
  \beamertemplateballitem
}

\usetheme{Antibes}
\usecolortheme{dolphin}

\usepackage[T1]{fontenc}
\beamertemplatetransparentcovered




\title
  	{{Анализ контуров.}}


 %\subject{Theoretical Computer Science}
 \date{Екатеринбург,~2015}


 \begin{document}
    \frame{\titlepage}

    \begin{frame}
        \frametitle{Введение}
        \uncover<+-> {Рассматриваемые в данной лекции контуры - это границы объектов.
        То есть замкнутые кривые без самопересечений.}
    \end{frame}

    \begin{frame}
        \begin{columns}[T]

            \begin{column}{.4\textwidth}
                \begin{columns}
                    \begin{column}{\textwidth}
                        \visible<2->{\includegraphics[width=\textwidth]{FirstPixel}}
                    \end{column}
                \end{columns}

                \begin{columns}
                    \begin{column}{\textwidth}
                        \visible<1->{\includegraphics[width=\textwidth]{contour_def}}
                    \end{column}
                \end{columns}
        	
            \end{column}

            \begin{column}{.1\textwidth}
        	    \visible<2->{\includegraphics[width=\textwidth]{ContourCoords}}
            \end{column}

            {\begin{column}{.5\textwidth}
                \only<1-1>{Интересующие нас объекты располагаются на пиксельной сетке, значит контур - последовательность пикселей.}
        	    \only<2->{В дальнейшем будет удобно представлять каждый пиксель контура не через абсолютные координаты,
                            а как сдвиг относительно предыдущего}
            \end{column}}
        \end{columns}
    \end{frame}

    \begin{frame}
        \frametitle{Почему Комплексные числа?}
        \only<1>
        {
            Рассмотрим произведение двух комплексных чисел: \newline $a+bi$ и $c-di$
            \newline
            $(a+bi)*(c-di) = (ac+bd) +(-ad+bc)i$
            \begin{columns}
                \begin{column}{.5\textwidth}
                    Действительная часть результата - в точности скалярное произведение векторов $(a,b)$ и $(c,d)$

                \end{column}
                \begin{column}{.5\textwidth}
                    Мнимая - скалярное произведение векторов $(a,b)$ и $(-d,c)$

                \end{column}
            \end{columns}
        }
        \only<2>
        {
            Зная скалярное произведение легко получить косинус угла между векторами.
            Если повернём первый вектор на 90$^{\circ}$, то косинус между получившимися векторами будет синусом угла поворота от второго вектора к исходному первому.
        }
        \only<3>
        {
            %$(ac+bd) +(-ad+bc)i$
            Таким образом, нормировав результат произведения, мы имеем в действительной части косинус, а в мнимой синус угла поворота от вектора $(c,d)$ к вектору $(a,b)$ 
        }
        
        \setlength{\unitlength}{2mm}
                \begin{picture}(0, -20)
                    \put(20, -10){\vector(1, 0){10}}
                    \put(30, -10){$(a,b)$}
                    \put(20, -10){\vector(2, 3){5}}
                    \put(25, -5){$(c,d)$}
                    \put(20, -10){\vector(-3, 2){5}}
                    \put(15, -5){$(-d, c)$}
                \end{picture}

    \end{frame}

    \begin{frame}
        
    \end{frame}

    \begin{frame}

    \end{frame}


    \begin{frame}
        \frametitle{ввдение}
        {\Huge текст}
        \begin{itemize}
          \item<+-> first
          \item<+-> second
          \item<+-> third

            \begin{enumerate}
              \item<1-> 1
              \item<1-2> \alert<5-> {2}
              \item<3-> third
            \end{enumerate}
        \end{itemize}
        \uncover<4-> {hi}
    \end{frame}

    \begin{frame}
        \includegraphics[width=.8\textwidth]{firstImage}
    \end{frame}

    \begin{frame}
    	\begin{columns}[T]
        	\begin{column}{.5\textwidth}
            	Your textblock
           	
        	\end{column}
        	\begin{column}{.5\textwidth}
            	\begin{block}{Your image}
                	\includegraphics[width=\textwidth]{firstImage}
            	\end{block}
        	\end{column}
    	\end{columns}
	\end{frame}

    \begin{frame}
        Improved integral $\int\limits_{a}^{b} x^2 dx$ inside text
    \end{frame}

\end{document} 